%% Generated by Sphinx.
\def\sphinxdocclass{report}
\documentclass[letterpaper,10pt,russian]{sphinxmanual}
\ifdefined\pdfpxdimen
   \let\sphinxpxdimen\pdfpxdimen\else\newdimen\sphinxpxdimen
\fi \sphinxpxdimen=49336sp\relax

\usepackage[margin=1in,marginparwidth=0.5in]{geometry}
\usepackage[utf8]{inputenc}
\ifdefined\DeclareUnicodeCharacter
  \DeclareUnicodeCharacter{00A0}{\nobreakspace}
\fi
\usepackage{cmap}
\usepackage[T1]{fontenc}
\usepackage{amsmath,amssymb,amstext}
\usepackage{babel}

\usepackage[Sonny]{fncychap}
\usepackage{longtable}
\usepackage{sphinx}

\usepackage{multirow}
\usepackage{eqparbox}

% Include hyperref last.
\usepackage{hyperref}
% Fix anchor placement for figures with captions.
\usepackage{hypcap}% it must be loaded after hyperref.
% Set up styles of URL: it should be placed after hyperref.
\urlstyle{same}
\addto\captionsrussian{\renewcommand{\contentsname}{Содержание:}}

\addto\captionsrussian{\renewcommand{\figurename}{Рис.}}
\addto\captionsrussian{\renewcommand{\tablename}{Таблица}}
\addto\captionsrussian{\renewcommand{\literalblockname}{Список}}

\addto\extrasrussian{\def\pageautorefname{страница}}

\setcounter{tocdepth}{1}



\title{SmartAn Documentation}
\date{февр. 04, 2017}
\release{}
\author{Yaroslav Sergienko}
\newcommand{\sphinxlogo}{}
\renewcommand{\releasename}{Выпуск}
\makeindex

\begin{document}

\maketitle
\sphinxtableofcontents
\phantomsection\label{\detokenize{index::doc}}



\chapter{Выдержки из положения о курсовой работе}
\label{\detokenize{coursework::doc}}\label{\detokenize{coursework:smartan}}\label{\detokenize{coursework:id1}}
\sphinxstyleemphasis{Программный проект} --- разработка программной системы или программно-аппаратного комплекса. Основным результатом программного проекта является законченная, готовая к использованию программная система или программно-аппаратный комплекс.


\section{Программный проект включает}
\label{\detokenize{coursework:id2}}\begin{itemize}
\item {} 
подготовку технического задания;

\item {} 
обзор и сравнительный анализ существующих решений;

\item {} 
подробное формализованное описание предлагаемого решения;

\item {} 
описание системы или технологии с точки зрения пользователя;

\item {} 
обоснование оптимальности выбранных решений, в том числе на основе экспериментальной оценки;

\item {} 
сравнение предлагаемой системы или технологии с известными аналогами по
\begin{itemize}
\item {} 
функциональности,

\item {} 
эффективности,

\item {} 
удобству использования;

\end{itemize}

\item {} 
демонстрацию системы;

\item {} 
документацию к ней.

\end{itemize}


\section{Критерии оценки}
\label{\detokenize{coursework:id3}}\begin{enumerate}
\item {} 
Полнота  достижения поставленных целей и задач работы

\item {} 
Полнота освещения состояния предметной области и использования источников информации

\item {} 
Сложность и объёмность программной  реализации или предложенных технологических решений

\item {} 
Качество итогового продукта, в т.ч. полнота верификации и тестирования, и т.д.

\item {} 
Качество оформления работы, в т.ч. отчёта и программного кода. Ясность и четкость изложения в отчёте

\item {} 
Четкость выдерживания запланированного графика работы, своевременность прохождения основных этапов выполнения КР, взаимодействие с руководителем КР

\end{enumerate}


\section{Сроки}
\label{\detokenize{coursework:id4}}
\sphinxstylestrong{До 15 февраля.}
Подготовка проекта КР и представление его руководителю. В проекте должны быть представлены:
\begin{itemize}
\item {} 
актуальность,

\item {} 
структура работы,

\item {} 
замысел,

\item {} 
список основных источников для выполнения данной работы,

\item {} 
ожидаемый результат

\end{itemize}

\sphinxstylestrong{До 13 апреля 2016.}
Изменение/уточнение темы КР, формата КР и смена руководителя КР.

\sphinxstylestrong{Не позднее 11 мая 2016.} Первое предъявление готовой КР руководителю. При необходимости студентом проводится доработка КР.


\chapter{Техническое задание}
\label{\detokenize{specification::doc}}\label{\detokenize{specification:id1}}

\chapter{Существующие решения}
\label{\detokenize{competitors::doc}}\label{\detokenize{competitors:id1}}

\chapter{Описание решения}
\label{\detokenize{solution::doc}}\label{\detokenize{solution:id1}}

\chapter{Описание системы с точки зрения пользователя}
\label{\detokenize{user_view::doc}}\label{\detokenize{user_view:id1}}

\chapter{Обоснование оптимальности выбранных решений}
\label{\detokenize{rationale::doc}}\label{\detokenize{rationale:id1}}

\chapter{Сравнительный анализ и известными решениями}
\label{\detokenize{comparison::doc}}\label{\detokenize{comparison:id1}}

\chapter{Демонстрация системы}
\label{\detokenize{demonstration::doc}}\label{\detokenize{demonstration:id1}}

\chapter{Файловая структура}
\label{\detokenize{files::doc}}\label{\detokenize{files:id1}}

\chapter{Интерфейс к загрузчику скриптов}
\label{\detokenize{script_interface::doc}}\label{\detokenize{script_interface:id1}}

\chapter{Арбитражный суд}
\label{\detokenize{arbitr::doc}}\label{\detokenize{arbitr:id1}}

\chapter{Парсер nalog.ru}
\label{\detokenize{nalog:nalog-ru}}\label{\detokenize{nalog::doc}}

\chapter{Сервис распознавания каптчи}
\label{\detokenize{captchatypers::doc}}\label{\detokenize{captchatypers:id1}}


\renewcommand{\indexname}{Алфавитный указатель}
\printindex
\end{document}